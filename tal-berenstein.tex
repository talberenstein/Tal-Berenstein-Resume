% LaTeX source of my resume
% =========================

% Commented for easy reuse... ;)

% See the `README.md` file for more info.

% This file is licensed under the CC-NC-ND Creative Commons license.


% Start a document with the here given default font size and paper size.
\documentclass[10pt,a4paper]{article}

% Set the page margins.
\usepackage[a4paper,margin=0.75in]{geometry}

% Setup the language.
\usepackage[english]{babel}
\hyphenation{Some-long-word}

% Makes resume-specific commands available.
\usepackage{resume}




\begin{document}  % begin the content of the document
\sloppy  % this to relax whitespacing in favour of straight margins


% title on top of the document
\maintitle{Tal Berenstein}{November 13, 1990}{Last update on \today}

\nobreakvspace{0.3em}  % add some page break averse vertical spacing

% \noindent prevents paragraph's first lines from indenting
% \mbox is used to obfuscate the email address
% \sbull is a spaced bullet
% \href well..
% \\ breaks the line into a new paragraph
\noindent\href{mailto:tal.berenstein@pm.me}{tal.berenstein\mbox{}@\mbox{}pm.me}\sbull
\textsmaller{+}46 765535845\sbull
\href{https://github.com/talberenstein}{github.com/talberenstein}\sbull
\href{https://linkedin.com/in/tali-berenstein-torres-94055726b/}{linkedin.com/tal-berenstein}
\\
Rambergsvägen 15B,
41713\thinspace {\large \sc }\sbull
Göteborg\sbull
Sweden

\spacedhrule{0.9em}{-0.4em}  % a horizontal line with some vertical spacing before and after

\roottitle{Summary}  % a root section title

\vspace{-1.3em}  % some vertical spacing
\begin{multicols}{2}  % open a multicolumn environment
\noindent \emph{Entrepreneurial geek with roots in the open-source movement. Passionately enabling software-related teams to deliver. Creates/\ improves processes trying to achieve the best possible result. Adaptive to both individuals and groups..}
\\
\\
At the age of six (1996), I got my first laptop in a humble house in \emph{Valparaíso, Chile}.  An old one with \acr{Windows 3.1} on a \acr{IBM I-Series} laptop. Spending hours playing with the MS-DOS shell back those years, my passion for coding had started. 11 years later (2007) I attended my first conference on an emerging new product, the new iPhone 3G, at the \emph{Universidad Diego Portales} in \emph{Santiago}.

After that conference I decided to study at the same University, motivated by the group of teachers and exposers at that conference. I taught myself a variety of skills, including system administration and programming (Java, Javascript, Ruby \& \CPP), along with the theoretical knowledge given by the program. In 2013 I was invited to a new Start-up with offices in Brazil. It was my first approach to SCRUM methodology.

Extensively travelled in Oceania and Asia when I started to feel I needed to also achieve a skill that I did not know I had: natural languages. I taught myself Portuguese, English, and currently learning Swedish. All those years I  worked as a freelancer for different projects and different groups of people and cultures. Currently leading a group of Front-end developers and designers for Volvo Group Connected Solutions and participating in a 6 months course of {\href{https://www.globalleadership.com/en/}{\acr{Global Functional Leaders}}} in Gothenburg, Sweden.



\end{multicols}


\spacedhrule{0em}{-0.4em}

\roottitle{Experience}

\headedsection
  {\href{https://www.volvogroup.com/en/about-us/organization/other-entities/volvo-group-connected-solutions.html}{\acr{Volvo Group Connected Solutions}}}
  {\textsc{Gothenburg, Sweden}} {%
  \headedsubsection
    { \acr{Front-end Application Engineer} }
    {Feb \apo21 -- present}
    {\bodytext{Leading and developing within an agile team a portal with technical tools and applications for a wide group of users to enable in-depth troubleshooting and data analysis across a complete stack of micro-services for connected solutions.
Developing, deploying, and operating a frontend stack for these new tools and applications according to development standards and clean architecture.

Source code management systems such as Gerrit GIT and GitHub,  micro-service architecture e.g. AWS cloud services, event-driven systems and web socket programming, CLEAN architecture and DRY programming.}}
}

\headedsection
  {\href{http://talberenstein.github.io}{\acr{Tal Berentein - Freelance.}}}
  {\textsc{Melbourne, Australia}} {%
  \headedsubsection
    {Freelance \acr{Full-Stack} Developer}
    {Oct \apo17 -- Jan\apo21t}
    {\bodytext{Provides services related to software development such as:\@ coaching, audits, process improvement, web/cloud architecture and stack strategies.  (Mostly but not only: NodeJS, GraphQL, React (Js and Native), Javascript, HTML5 and CSS, PHP). Also worked with SQL and NoSQL databases.
  \\ Clients:\@ \href{http://www.happycamperpizza.com.au}{HappyCamperPizza.com.au}, \href{https://delightchile.cl}{DelightChile.cl} \& \href{https://www.orakelforlag.se}{OrakelFörlag.se}.}}
}

\headedsection
  {\href{http://www.gestsol.cl}{Gestsol}}
  {\textsc{Santiago, Chile}} {%
  \headedsubsection
    {Front-end Developer}
    {Apr \apo15-- Oct \apo16}
    {\bodytext{Front-end mobile and web developer using technologies such as: Ionic, Angular 1.x, CoffeeScript and Ruby on Rails 5.0. Mobile application similar to Netflix, fetching data from PHP endpoints to show in a video player developed by the team. Web Application GPS Real-Time application on CoffeeScript and Ruby on Rails fetching data and creating models from/to API developed in Elixir Erlang.}}
}


\vspace{-0.2em}
\begin{center}
  \emph{\small Please refer to \href{http://www.linkedin.com/in/tal-berenstein}{Tal's LinkedIn profile} for a more complete list of work experience along with recommendations.}
\end{center}


\spacedhrule{-0.2em}{-0.4em}

\roottitle{Education}

\headedsection
  {\href{http://www.udp.cl}{Diego Portales University}}
  {\textsc{Santiago, Chile}} {%
  \headedsubsection
    {BSc in Computer Science \textnormal{~(\acr{CS}~\,\&\,Telecomunications)}}
    {2009 -- 2013}
    
}


\spacedhrule{0.5em}{-0.4em}

\roottitle{Another Skills}

\inlineheadsection  % special section that has an inline header with a 'hanging' paragraph
  {Teaching experience:}
  {  Currently, I am teaching a small group of junior developers about React, Typescript and CLEAN architecture, so they can get their first job in this passionated field. }

\vspace{0.5em}
\inlineheadsection
  {Natural languages:}
  {Spanish \emph{(mother tongue)}, English \emph{(full professional proficiency)}, Swedish \emph{(limited working proficiency)} and Portuguese \emph{(limited working proficiency)} }


\spacedhrule{1.6em}{-0.4em}

\roottitle{Interests}

\inlineheadsection
  {Non-exhaustive and in alphabetical order:}
  {Art, Buddhism, chess, cryptography, functional programming, history, music (from classical and jazz to Berlin-techno), philosophy, poetry,  startups, travel, typography (e.g. \LaTeX), \acr{UX}-design, vegetarian cuisine and  \acr{VRP (Vehicle Routing Problem)}.}


\end{document}
